\documentclass[10pt,a4paper,oneside]{article}
\usepackage[utf8]{inputenc}
\usepackage{amsmath}
\usepackage[T1,T2A]{fontenc}
\usepackage[utf8]{inputenc}
\usepackage[english, russian]{babel}%Подключаем русский язык.
\usepackage{graphicx}
\graphicspath{{pictures/}}
\RequirePackage{float}
\usepackage[left=2cm,right=2cm,top=2cm,bottom=2cm]{geometry} % Меняем поля страницы.
\usepackage{listings}
\lstdefinestyle{Style}{
	numbersep=10pt,
	frame=single,
	captionpos=b
}

\begin{document}
	\begin{titlepage}
		\newpage
		\begin{center} % Размещение ткста - по центру
			Санкт-Петербургский политехнический\\ 
			университет Петра Великого\\
			Институт компьютерных наук и технологий\\
			Кафедра компьютерных систем и программных технологий\\
			\vspace{7cm}
			\textbf {Отчёт по лабораторной работе}\\
			\textbf {Дисциплина:} Телекоммуникационные технологии\\
			\textbf{Тема:} Реализация TCP-сервера на UNIX системе и клиента на Windows
		\end{center} % Конец размещения
		\vspace{8cm} % 
		
		\vfill
		
		\flushleft{Выполнил студент группы 33501/1} 
		\hfill\parbox{9 cm}{\hspace*{3cm}\hbox to 0cm{\raisebox{-1em}{\small(подпись)}}\hspace*{-0.8cm}\rule{3cm}{0.8pt}Чурляев Д.О.}\\[0.6cm]
		
		\flushleft{Преподаватель} \hfill\parbox{9 cm}{\hspace*{3cm}\hbox to 0cm{\raisebox{-1em}{\small(подпись)}}\hspace*{-0.8cm}\rule{3cm}{0.8pt} Алексюк А. }\\[0.6cm]
		
		\hfill\parbox{9 cm}{\hspace*{5cm} \today }\\[0.6cm]
		\vspace{\fill}
		\begin{center}
			Санкт-Петербург \\ 2017
		\end{center}
	\end{titlepage}
	
\newpage
\section{Цель работы}

Разработать клиент-серверное приложение «Система выставления курсов валют».\\
Платформа сервера: UNIX\\
Платформа клиента: Windows\\

\section{Реализованные функции сервера:}

\begin{itemize}
\itemПрослушивание определенного порта 
\itemОбработка запросов на подключение по этому порту клиентов 
\itemПоддержка одновременной работы нескольких клиентов с использованием механизма нитей и средств синхронизации доступа к разделяемым между нитями ресурсам. 
\itemПринудительное отключение конкретного клиента 
\itemДобавление новой валюты (кода валюты) 
\itemУдаление валюты 
\itemДобавление курса конкретной валюты 
\itemВыдача пользователю списка имеющихся валют с текущими курсами и абсолютными/относительными приращениями к предыдущим значениям 
\itemВыдача пользователю истории курса конкретной валюты 
\item* Сохранение состояния при выключении сервера
\end{itemize}


\section{Реализованные функции клиента:}


\begin{itemize}
\itemВозможность параллельной работы нескольких клиентов с одного или нескольких IP-адресов 
\itemУстановление соединения с сервером (возможно, с регистрацией на сервере)
\itemРазрыв соединения 
\itemОбработка ситуации отключения сервером 
\itemПолучение и вывод списка валют с котировками/изменениями 
\itemПередача команды на добавление валюты 
\itemПередача команды на удаление валюты 
\itemПередача команды на добавление курса валюты 
\itemПолучение и вывод истории котировок валюты 
\end{itemize}
\newpage

\section{Формат команды:}

Команды представляют из себя текстовые сообщения, в которых команда и ключ разделены пробелом, если ключ необходим. В клиентских команда перед самой командой необходим символ "/".\\


\section{Команды сервера:}
help: print this help message\\
list: list connected clients\\
kill [id]: disconnect client with specifie\\
killall: disconnect all clients\\
shutdown: shutdown server\\


\section{Команды клиента}
/add [currency] : add new currency\\
/addv [currency] [value] : add new value to currency\\
/del [currency] : remove currency\\
/all : list all currencies\\
/hist [currency] : history for currency\\


\section{Описание:}
Для запуска сервера, необходимо сначала включить базу данных, затем сам сервер. Далее клиенты могут подключиться к серверу. Всего в данной реализации на серверном сокете 4 обработчика (workers) поэтому подключиться могут 4 клиента.\\
Обработчики из себя представляют объекты thread\_pool. По сути это потоки для обработки клиентов.\\
Для реализования многопоточности используеться epoll или мультеплексировнный вход. epoll обрабатывает сокет и говорит когда им можно пользоваться обработчикам\\

Тип сокета:

socket(AF\_INET, SOCK\_STREAM, 0)

\end{document}













